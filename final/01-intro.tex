\section{Introducción}

La relación de series de tiempo puede abordarse desde distintas metodologías, como los modelos autorregresivos vectoriales, regresiones lineales múltiples \cite{Wei2019}, modelos estocásticos que incluyen correlaciones cruzadas o análisis de frecuencias de Fourier \cite{Pandit2001}, modelos estadísticos de causación \cite{Eichler2013} entre muchos otros procedimientos de análisis multivariado \cite{Wei2019a}. Todos estos modelos requieren de ciertas consideraciones de las series de tiempo que se comparan, como por ejemplo que se trate de series de tiempo estacionarias. El aprendizaje por refuerzo representa una manera de encontrar variables de salida mediante el aprendizaje de variables de entrada. Es posible convertir las series de tiempo univariadas o multivariadas a estructuras capaces de ser analizadas mediante algoritmos de aprendizaje por refuerzo. En este reporte, se mostrará la manera de hacer esta conversión. 